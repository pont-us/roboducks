\section{Genetic training}
\label{theory-ga}

Genetic algorithms are conceptually simple, but are widely applicable
and can prove highly effective. The method is modelled on the
principle of natural selection. It can be employed on virtually any
problem where (1) the parameters of a possible solution can be encoded
as a relatively compact sequence of numbers or symbols and (2) there
exists a \emph{fitness function} or \emph{objective function} which
can assign a numerical value reflecting how ``good'' a candidate
solution is.\footnote{The satisfaction of these two criteria does not,
  of course, guarantee that a particular genetic algorithm will work!}

Genetic algorithms come in numerous variations, but given a suitable
problem the general procedure is as follows\footnote{This summary is
  based on the descriptions given in \cite{wasserman93} and 
  \cite{winston92}.}:

\begin{enumerate}

\item Devise a way of encoding the parameters of a solution into a
  fixed-length string or numbers or symbols. Very often, binary digits
  are used, but this is by no means mandatory. The choice of encoding
  method can have a huge impact on the performance of the algorithm.

\item Initialize a set of randomly generated strings, called the
  \emph{population} for obvious reasons.

\item Repeatedly apply the following three techniques:

  \begin{enumerate}
    
  \item \emph{Reproduction}: Assess the fitness of each individual in
    the population, and produce a ``weighted copy'' of the current
    population. That is, initialize a new population of the same size
    by, for each string, copying a string selected at random from the
    old population. The selection is linearly weighted by fitness, so
    that fitter strings have a higher chance of being selected.

  \item \emph{Crossover}: select pairs of strings from the
    population. For each pair, randomly select a point within the
    length of a string. Break both strings at this point, and exchange
    their final segments. For example, \texttt{001122} and
    \texttt{222333} might be split and recombined halfway to give
    \texttt{001333} and \texttt{222122}.
    
  \item \emph{Mutation}: perturb one or more strings. A string is
    selected at random, an element selected at random within that
    string, and a random modification applied to that element.
    Traditionally this method is used very sparingly; mutation rates
    of the order of one in a thousand are common.
 
  \end{enumerate}

\end{enumerate}

The genetic algorithm used in this project deviates somewhat from this
standard outline; the differences are described in Section
\ref{theory-ga-nn}.
